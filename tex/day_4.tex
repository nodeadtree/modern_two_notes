\documentclass[../notes.tex]{subfiles}
\begin{document}
\section{Day 3}
\subsection{Modular Arithmetic}
\begin{center}
    \[
        \faktor{\Z}{n\Z}
    \]
\end{center}
\subsubsection{Quotient structures:}
\begin{enumerate}
    \item $X$ is a set, equipped with an equivalence relation
        $\sim$.
        \begin{align*}
            X \rightarrow \faktor{X}{\sim}
        \end{align*}
    \item $R$ a ring, $I\in R$, where $I$ is an ideal.
        \begin{align*}
            R,I\ \rightarrow \faktor{R}{I}
        \end{align*}
    \item $G$ a group, $H \nabla G$ a normal subgroup
        \begin{align*}
            G, H\rightarrow \faktor{G}{H}
        \end{align*}
\end{enumerate}
Note that 1 specializes to 2 and 3\\
\vspace{.25cm}
$X$, $\sim$ a relation $\equiv$ a set of ordered pairs
of elements of $X$.
\underline{Notation:}
\[
    x\sim y\equiv (x,y)\text{ is in the set of ordered pairs }
\]
\begin{definition}
    $\sim$ is an equivalence relation if the folowing conditions
    are satisfied
    \begin{enumerate}
        \item $x\sim x$
        \item $x\sim y \implies y\sim x$
        \item $x \sim y\sim z\implies x\sim
            z$
    \end{enumerate}
\end{definition}
\begin{itemize}
    \item \underline{Example:} Discrete equivalence relationship, 
        $x\sim y \iff x=y$
    \item \underline{Example:} $\forall x,y,\ x\sim y$
\end{itemize}
\begin{definition}
    \begin{enumerate}
        \item $(x,\sim)$ is an equivalence relation. $x\in X$ the
            equivalence class of $x$ denoted $\hat{x}$ is the set of $y$ with
            $x\sim y$
        \item $X$ breaks up into the (non-overlapping) equivalence class.
            \begin{center}
                \[
                    z\in \overline{x}\cap\overline{y}
                \]
                \Large{CIRCLE WITH OTHER CIRCLES IN IT}
            \end{center}
    \end{enumerate}
\end{definition}
\begin{definition}
    \begin{align*}
        \faktor{\Z}{n\Z}\text{ (the set of equivalence classes) }\\\\
    \end{align*}
    With the set of equivalence classes as $\{\overline{0}, \overline{1},
    \dots, \overline{n-1}\}$
    \begin{align*}
        n\in \N&=\{1,2,\dots\}\\
        a\sim b &\equiv a=b(mod n) \iff n|a-b\\
    \end{align*}
    \begin{enumerate}
        \item There is an equivalence relation
        \item $a, a^{\prime},b, b^{\prime}\in \Z$
    \end{enumerate}
\end{definition}
\begin{align*}
    \overline{a}=\overline{a}^{\prime}\\
    \overline{b}=\overline{b}^{\prime}\\
    \implies \overline{a+b}=\overline{a^{\prime}+b^{\prime}}
    \implies \overline{ab}=\overline{a^{\prime}b^{\prime}}
\end{align*}
To show $\overline{a+b}=\overline{a^{\prime}+b^{\prime}}$,
\begin{align*}
    n|(a+b)-(a^{\prime}+b^{\prime})\\
    n|(a-a^{\prime})+(b-b^{\prime})\\
\end{align*}
To show $\overline{ab}=\overline{a^{\prime}b^{\prime}}$
\begin{align*}
    n|(ab-a^{\prime}b^{\prime})\\
    ab-ab^{\prime}+ab^{\prime}-a^{\prime}b^{\prime}\\
    n|a(b-b^{\prime})+(a-a^{\prime})b^{\prime}
\end{align*}
\begin{align*}
    \overline{0}&=n\Z\\
    \overline{1}&=1+n\Z\\
    \overline{2}&=2+n\Z\\
    &\vdots\\
    \overline{n-1}&=(n-1)+n
\end{align*}
\begin{definition}
    $\faktor{\Z}{2\Z}$ is a ring w.r.t,
    \begin{align*}
        \overline{a}\cdot\overline{b}\equiv\overline{ab}\\
        \overline{a}+\overline{b}\equiv\overline{a+b}
    \end{align*}
    \begin{enumerate}
        \item
            \vspace{.1cm}
            \begin{align*}
                (\overline{a}+\overline{b})+\overline{c}=
                \overline{a}+(\overline{b}+\overline{c})\\
                (\overline{a}+\overline{b})+\overline{c}
                =\overline{a+b}+\overline{c}=
                \overline{(a+b)+c}\\
                \overline{a}+(\overline{b}+\overline{c})=
                \overline{a}+\overline{b+a}=\overline{a+(b+c)}
            \end{align*}
        \item
            \vspace{.1cm}
            \begin{align*}
                (\overline{a}\overline{b})\overline{c}=
                \overline{a}(\overline{b}\overline{c})
            \end{align*}
        \item
            \vspace{.1cm}
            \begin{align*}
                \overline{a}=\overline{a}+\overline{0}=
                \overline{0}+\overline{a}=\overline{a}
            \end{align*}
        \item
            \vspace{.1cm}
            \begin{align*}
                \overline{a}+\overline{x}=\overline{0}
                \iff \overline{x}=\overline{-a}
            \end{align*}
        \item
            \vspace{.1cm}
            \begin{align*}
                \overline{a}+\overline{b}=\overline{b}+\overline{c}=\overline{a+b}
            \end{align*}
        \item
            \vspace{.1cm}
            \begin{align*}
                \overline{a}(\overline{b}+\overline{b})=
                \overline{a}\overline{b+c}\\
                =\overline{a(b+c)}\\
                =\overline{ab+ac}\\
                =\overline{ab}+\overline{ac}\\
                =\overline{a}\overline{b}+\overline{a}\overline{c}
            \end{align*}
        \item
            \vspace{.1cm}
            \begin{align*}
                \overline{1}\cdot\overline{a}=\overline{1\cdot a}=\overline{a}
            \end{align*}
    \end{enumerate}
\end{definition}
\begin{theorem}
    \underline{(This should be a proposition)}
    $\faktor{\Z}/{n\Z}$ is a field if and only if $n=p$ where
    $p$ is prime.
\end{theorem}
\begin{theorem}
    \underline{(This should be a proposition)}
    $F$ is a field if $F$ has no zero-divisors, or
    \begin{align*}
        ab=0\\
        \implies a=0 \text{ or } b=0\\
    \end{align*}
\end{theorem}
\begin{proof}
    Assume $ab=0$ and $a\neq0$
    \begin{align*}
        0=a^{-1}0=a^{-1}(ab)=(a^{-1}a)b=1\cdot b=b
    \end{align*}
\end{proof}
\begin{theorem}
    \underline{This should be a fact, also it confuses me}
    For every $p$ (a prime), and $n\in\N$, there is a \underline{unique}
    field of size $p^n$,
    \begin{align*}
        \mathbb{F}_{p^m}\le \mathbb{F}_{p^n} \text{ if } m|n
    \end{align*}
\end{theorem}
\end{document}
