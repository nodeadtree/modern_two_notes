\documentclass[../notes.tex]{subfiles}
\begin{document}
\section{Day 4}
\begin{align*}
    \faktor{\Z}{n\Z}=\{\overline{0},\overline{1},\overline{2},\dots,\overline{n-1}\}\\
    a\sim b \iff a=bmod(n)\\
\end{align*}
$\faktor{\Z}{n\Z}$ is a ring wrt,
\begin{align*}
    \overline{a}+\overline{b}=\overline{a+b}\\
    \overline{a}\cdot\overline{b}=\overline{ab}
\end{align*}
\begin{itemize}
    \item $\overline{0} $ $0$-element
    \item $\overline{1}$ is $1$ in $\faktor{\Z}{n\Z}$
\end{itemize}
\begin{theorem}
    $\faktor{\Z}{n\Z}$ is a field if and only f $n=p$ where $p$ is prime.
\end{theorem}
\begin{proposition}
    No field has $0$-divisors ($ab=0$ implies $a=0$ or $b=0$). Proof was
    supplied last class
\end{proposition}
\underline{Note:}
$R$ is a ring, therefore
\begin{align*}
    0\cdot r=0\\
    (-a)b=a(-b)=-(ab)
\end{align*}
$0$ and $1$ are unique.
\begin{proof}
$p$ does not divide $a$ if and only if $gcd(a,p)=1$.
\begin{align*}
    \exists x,y \in \Z,\ xa+yp=1\\
    \overline{xa+yp}=\overline{1}\\
    \overline{x}\cdot\overline{a}+\underbrace{\overline{y}\cdot\overline{p}}_{\text{cancels out for some reason}}=\overline{1}\\
    \overline{x}\cdot\overline{a}=\overline{1}\\
    \implies \faktor{\Z}{n\Z},\ \text{ a field }\\
\end{align*}
Assume $n$ is not a prime, where
\begin{align*}
    n=ab,\ \text{ with }\ a,b>1\\
    \overline{n}=\overline{a}\cdot\overline{b}\\
    \overline{0}=\overline{a}\cdot\overline{b}
\end{align*}
$\implies\faktor{\Z}{n\Z}$ has $0$-divisors because $\overline{a},
\overline{b}\neq 0$, thus the previous proposition is proven.
\end{proof}
\begin{definition}
    Given a ring $R$, we use the notation, $R^*$ to
    indicate that it has invertible elements. Furthermore, we observe that
    $R^*$ is an abelian group w.r.t. product.
\end{definition}
\begin{theorem}
    (This should be a corrolary). $p$ is a prime.
    \[
        (\faktor{\Z}{p\Z})^*=\{\overline{1},\overline{2},\dots,\overline{p-1}\}
    \]
\end{theorem}
\begin{theorem}
    (Also a corrolary)\\
    $\forall a\in \Z$ $p$ does not divide $a$.
    \[
        a^{p-1}=1mod(p)
    \]
\end{theorem}
\underline{Example:}
\begin{align*}
    p=3,\ a=16\\
    16^2=1mod(3)
\end{align*}
\begin{proposition}
    $G$ is a finite group.
    \begin{align*}
        \forall x\in G,\ x^{|G|}=1\\
        \overline{a}^{p-1}=\overline{1}\\
        a^{p-1}=1mod(p)
    \end{align*}
\end{proposition}
Alternative proof of Corrolary 2,
\begin{proof}
    \begin{align*}
        1\cdot 2\cdot 3\dots\cdot(p-1)\\
        x\mapsto axmod(p)\\
    \end{align*}
    Assume $ax=aymod(p)$. We want to show,
    \begin{align*}
        x=y\\
        \overline{a}\cdot\overline{x}= \overline{a}\cdot\overline{y}\\
        \overline{a}(\overline{x}-\overline{y})=0\\
        \overline{x}-\overline{y}=0
    \end{align*}
    Somehow, we move on to,
    \begin{align*}
        \overline{1} \cdot \overline{2} \cdot \overline{3}
        \cdot\dots
        (\overline{p-1})=
        (\overline{a\cdot 1})
        \cdot
        (\overline{a\cdot 2})
        \cdot\dots
        (\overline{a\cdot (p-1)})\\
        \overline{(p-1)}!=(\overline{a})^(p-1)\cdot\overline{(p-1)}!\\
        \overline{(p-1)}=\overline{z}\neq{0}\\
        \overline{z}=(\overline{a})^{p-1}\cdot\overline{z}\\
        \overline{z}-\overline{a}^{p-1}\cdot\overline{z}=0\\
        \overline{z}(\overline{1}-\overline{a}^{p-1})=0\\
    \end{align*}
\end{proof}
\subsection{Primality test}
$n$ Pick a coprime with $n$, check whether $a^{n-1}=1mod(n)$,
if they are not equal, then $n$ is not prime. Plus some other easily checked
conditions. This also somehow implies a fast algorithm for testing natural
numbers for primality.\\
Unfortunately (hahaha what? is there anything more unfortunate than the
previous factoid? in any case, I continue, with regret).
\begin{align*}
    \exists n\in \N\\
    a^{n-1}=1mod(n)\\
\end{align*}
for every $a$ with $gcd(a,n)=1$. Sick.\\
\begin{definition}
    \begin{align*}
        \varphi(n)=|\{i:1\leq i\leq n-1,\ gcd(i,n)=1\}|\\
        \varphi(p)=p-1\\
        \varphi(n)=
        n(1-\frac{1}{p_1}) \dots (1-\frac{1}{p_k})\\
    \end{align*}
    $p_1,\dots,p_k$ are the prime factors of $n$
\end{definition}
\begin{theorem}
    \begin{align*}
        a\n\in N,\ gcd(a,n)=1\\
        a^{\varphi(n)}=1mod(n)\\
    \end{align*}
\end{theorem}
\begin{proposition}
    (Should be an idea. Utter shrapnel of an idea, but an idea nonetheless)
    \[
        (\faktor{\Z}{n\Z})=\{\overline{i}: 1\leq i\leq n-1,\ gcd(i,n)=1\}
    \]
\end{proposition}
\begin{theorem}
    (Should be a lemma)\\
    \[
        \varphi(ab)=\varphi(a)\varphi(b)
    \]
    if $a,b\in \N$ with $gcd(a,b)=1$
    \begin{align*}
        n=p_1^{d_1}\dots p_k^{d_k}
        \varphi(p^d)=
    \end{align*}
    Yes, that lemma ends there. That's it. No more lemmas.
\end{theorem}
\end{document}
